\documentclass{llncs}

\usepackage{amssymb}
\usepackage{proof}
%\usepackage{hyperref}
\usepackage{url}
\usepackage{amsmath}
\usepackage{cite}
\usepackage{latexsym, graphics, graphicx}
\usepackage{color}
\usepackage{epsfig, amssymb}
\usepackage{yfonts}
\usepackage{algorithmic}
\usepackage{algorithm}
\usepackage{float}
\usepackage{enumerate}
\usepackage{hyperref}
\usepackage{marginnote}
\usepackage{mathtools}
\usepackage{todonotes}



\newtheorem{assumption}{Assumption}

\newcommand{\pa}{{\bf Pair}}
\newcommand{\enc}{{\bf Enc}}
\newcommand{\eq}{\mathcal{EQ}}
\newcommand{\bphi}{\ovr{\phi}}
\newcommand{\bh}{\ovr{h}}
\newcommand{\tdh}{\tilde{h}}
\newcommand{\ignore}[1]{}
\newcommand*\rfrac[2]{{}^{#1}\!/_{#2}}

%\newcommand{\ISFULL}

\ifdefined\ISFULL
\newcommand{\FULL}[1]{#1}
\newcommand{\SHORT}[1]{}
\else
\newcommand{\FULL}[1]{}
\newcommand{\SHORT}[1]{#1}
\fi



\title{ $MOE$ Decision Problems}


\pagestyle{empty}

\begin{document}

\maketitle
\begin{abstract}
The $MOE$ decision problem. 

\end{abstract}

\section{Definitions}

\begin{definition}
	$MOE$-terms: Given a first-order signature $\Sigma$, a countable set of (free) constants (names) $N$, such that $\Sigma \cap N =\emptyset$, and a (countable) set of variables $X$, the set of $MOE$-terms over $X$ is denoted by $T(\Sigma \cup N,X)$. 
\end{definition}



\begin{definition}
	$MOE$-Program: Is an interaction between the 
	the adversary and the oracle consisting of the following actions
	and orders. Let $c$ be a name representing a session identifier.
	\begin{enumerate}
		\item $start(c)$. The start of a session $c$, sent from the
		adversary to oracle.
		\item $stop(c)$. The end of session $c$.
		\item $block(c, x)$. A new block in session $c$ sent from
		the adversary to the oracle. This is to be encrypted by the 
		oracle. This variable, $x$, represents possible messages, via
		instantiation, that the adversary could send during a program
		run. 
		\item $send(c, m)$. The oracle send a $MOE$-term, $m$, back
		to the adversary. The variables of $m$ must have been 
		previously contained in a received $block(c, x)$ message
		that occurred before the $send(c, m)$ message.
	\end{enumerate} 
\end{definition}
This ordering in the last item will be important to the decision 
problems. We can more compactly represent the sequence of messages
in a $MOE$-program as a frame. 

\begin{definition}
		$MOE$-frame:  $\phi = \nu \tilde{n}.\sigma$, where
		$\tilde{n}$ is a set of names representing non-free variables such as random nonces. $\sigma$ is a substitution
		such that $Dom(\sigma) = \{ y_1, y_2, \ldots, y_n\}$,
		$Dom(\sigma) \cap \tilde{n} = \emptyset$, and
		$Ran(\sigma)$ are $MOE$-terms.  
\end{definition}

A $MOE$-frame models a trace of a $MOE$ program in the following way:
if the $i$th action of the challenger in the trace is 
$Rcv\_Block(t)$ or $Send(t)$, where $t$ is a $MOE$-term, 
then there is a variable $y_i \in Dom(\sigma)$ and 
$y_i \sigma \mapsto t$. Random variables such as the IV are 
modeled as names and placed in $\tilde{n}$.

The terms of the frame are further restricted depending on the mode of encryption being modeled and the schedule used by the oracle to
return cipher blocks to the adversary. These restrictions have
the following impact on the $MOE$-frames:
\begin{itemize}
	\item $MOE$: The mode of encryption dictates how the oracle 
	constructs cypher blocks. For example, in a CyberBlock Chaining,
	$CBC$, $MOE$ the ith block of cipher text, $C_i$, is modeled by
	the $y_i \mapsto f(C_{i-1} \oplus x_i)$, where $\oplus$ is
	the xor function.
	\item Schedule: The oracle can either return the 
	cipher blocks immediately or at the end of the session.  
\end{itemize}

The adversary has the ability to execute multiple simultaneous session with the oracle. In this case the initial name, the IV, will be fresh for each session. Each session can then be modeled by it's own frame. 


%\begin{definition}
%	Let 
%\end{definition}


\section{$MOE$ Decision Problems}

We model a potential attack via a decision problem involving unification. 

First we need a new definition of unification which includes the 
constraint on term orders imposed by the $MOE$-program. This 
ordering can be enforeced using a relation $\prec_{\mathcal{P}}$, defined 
as follows. 

\begin{definition}
	Let $\mathcal{P}$ be a $MOE$-program and $\phi = \nu \tilde{n}.\sigma$ the corresponding frame. We define a relation, 
	$\prec_{\mathcal{P}}$, by:
	\begin{itemize}
		\item $t \prec_{\mathcal{P}} x$,
		$y_i \mapsto x, ~y_j \mapsto t ~ \in \sigma$, and
		$ i < j$.
		\item $x_i \prec_{\mathcal{P}} x_j$ iff 
		 $y_i \mapsto x_i, ~y_j \mapsto x_j ~ \in \sigma$, and
		 $ i < j$.
		 \item If $\bar{t} \prec_{\mathcal{P}} x$, then 
		 $f(\bar{t}) \prec_{\mathcal{P}} x$, where $f \in \Sigma$.
		 \item $t \not \prec_{\mathcal{P}} x$ iff $t \prec_{\mathcal{P}} x$ does not hold.
	\end{itemize}
\end{definition}

We can now define the $MOE$ unification problem.

\begin{definition}
	Let $\mathcal{P}$ be a $MOE$-program and $\phi = \nu \tilde{n}.\sigma$ the corresponding frame. Let $t_1$ and $t_2$
	be two $MOE$-terms appearing in $\mathcal{P}$. $t_1$ and $t_2$
	are $\mathcal{P}$-unifiable iff there exists some substitution
	$\delta$ s.t.
	\begin{itemize}
		\item $\forall x \in Dom(\delta), ~x\delta \prec_{\mathcal{P}} x$.
		\item $t_1 \delta =_E t_2\delta$.
	\end{itemize}
\end{definition}


\begin{definition}\todo[inline]{Andrew: We need to improve this def}
	Decision Problem: We model an attack by the adversary being able
	to unify two terms, either from the same frame of from different
	frames.  
\end{definition}

Now that the general decision problem is defined, we can define several
instances of the problem based on the combination of the following factors:
\begin{itemize}
	\item The equational Theory $E$.
	\item The $MOE$.
	\item The Schedule.
	\item Bounds on the session length and number of sessions.  
\end{itemize}

We consider several of these possibilities in what follows.

First, we will consider the $E=xor$ equational theory and we 
will consider the following $MOE$. In these modes of encryption 
$\Sigma ={\oplus, 0, f}$, where $\oplus$ is xor and $f=enc(K,\_)$.
$IV$ is the initialization vector, the first block sent. 
\begin{itemize}
	\item Cipher Block Chaining $(CBC):$ The ith plain text is modeled via a term consisting of a single variable $x_i$. The initial cipher block, the IV, is modeled by a name, $r$. The ith block of cipher text, $C_i$, is modeled by a
	term $f(C_{i-1} \oplus x_i)$.
	\item Cipher Feedback $(CFB):$ The ith plain text is modeled via a term consisting of a single variable $x_i$. The initial cipher block is modeled by a name, $r$. The ith block of cipher text, $C_i$, is modeled by a
	term $f(C_{i-1}) \oplus x_i$.  
	\item Propagating Cipher Block Chaining ($PCBC$): 
	$C_1 = f(P_1 \oplus IV)$, $C_i = f(P_i \oplus P_{i-1} \oplus C_{i-1})$.
\end{itemize}

\subsection{$MOE_A$ Decision Problems} 
\begin{conjecture}
	Given a signature $\Sigma=\{
	\oplus, f\}$, where $\oplus$ is associative, the $CBC$ MOE, the immediate  schedule, unbounded sessions, and unbounded session 
	length.  The $MOE_A$ decision problem is undecidable.
\end{conjecture}\todo[inline]{Andrew: This is really just a proof sketch and we will need to make sure there isn't a problem.}
\begin{proof}
	Reduction from the PCP. Since $\oplus$ is just $A$ we can directly
	model string by chaining the symbols together via $\oplus$.
	
	For a PCP the adversary starts two classes of sessions. 
	The first class consists of a session for each of the string
	in the upper part of the PCP problem, $A_1, A_2, \ldots, A_n$.
	The second class consists of a session for each of the strings
	in the lower part of the PCP problems, $B_1, B_2, \ldots, B_n$.
	
	The adversary simulates each possible index (and thus possible PCP
	solution) by creating a session for each index in each class, A and B. To ensure each session in A are only unified with index in B
	the adversary uses a unique IV for each index in A that matches the same index in B. 
	
	Since there is in bound on the session length
	or the number of sessions if there is a solution, the index of that
	solution in the A set of index will be unified with the same index in the B set. Since unification is modulo $A$ there is then a solution to the PCP problem.  
\end{proof}

\subsection{$MOE_{\oplus}$ Decision Problems}
We will first consider the xor equational theory. We can therefore refine the definition of $MOE$-term 
 
\begin{definition}
	$MOE_{\oplus}$-terms: Given a first-order signature $\Sigma=\{
	\oplus, 0, f\}$, a countable set of (free) constants (names) $N$, such that $\Sigma \cap N =\emptyset$, and a (countable) set of variables $X$, the set of $MOE_{\oplus}$-terms over $X$ is denoted by $T(\Sigma \cup N,X)$. 
\end{definition}

\subsubsection{Finite Sessions of Finite Length}
The first decision problem to consider is the $MOE_{\oplus}$
problem with a finite bound on both the number of sessions and
the sessions lengths. 
For both schedules and both $MOE$s, $CBC$ and $CFB$, the problem
reduces to the $MOE$-unification problem for $E=xor$. This problem
has been solved in the positive [Chris and Hia Paper].

\subsubsection{Arbitrary Sessions of Arbitrary Length}
The next problem to consider is the opposite extreme, that of unbounded
number of sessions and an unbounded session length. 

\begin{conjecture}
The $MOE_{\oplus}$ decision problem is undecidable for unbounded 
sessions, unbounded session lengths, and immediate schedule. 
\end{conjecture}

\noindent
\textbf{some proof ideas}:\\
We would essentially like to reply the idea from the case where 
$E = A$. However, the difficulty is that $\oplus$ is $AC$ and thus
commutativity could result in false solutions. We can possibly 
get around this problem by using repeated applications of $f()$
to encode the position of a symbol in a string, i.e., 
$f(f(f(f(a))))$ would represent the the fact that `a' is the 
4th symbol in the string. 

\begin{example}
	consider the following PCP:
	\[
	(\frac{ba}{baa}),~(\frac{ab}{ba}),~(\frac{aaa}{aa})
	\]
	The solution for this basic PCP problem is $1,3$.
	The encoding for this index for the upper string of the PCP is
	\begin{align}
		c_1 &= f(f(b) \oplus IV_{1})\\
		c_2 &= f(f(f(a)) \oplus c_1)\\
		&\ldots\\
		c_5 &= f(f^5(a) \oplus c_4)
	\end{align}
	and the corresponding index in the set of lower string is
	\begin{align}
	c_1 &= f(f(b) \oplus IV_{1})\\
	c_2 &= f(f(f(a)) \oplus c_1)\\
	&\ldots\\
	c_5 &= f(f^5(a) \oplus c_4)
	\end{align}
	The same an thus are unifiable. However notice that the string 
	for index $abaaa$ for index $2,3$, has the same number of symbols,
	5, and the same makeup, a single `b' and four `a's. This string will not be unifiable with the solution $1,3$, since it's
	encoding is $f^1(a)f^2(b)f^3(a)f^4(a)f^5(a)$ which is not
	unifiable with $f^1(b)f^2(a)f^3(a)f^4(a)f^5(a)$. 
\end{example}\todo[inline]{Andrew: This is the current idea, does it
work or lead to other problems?}






\subsubsection{Finite Sessions of Arbitrary Length}
The next decision problems to consider are for $MOE_{\oplus}$ terms,
both an immediate and a delayed schedule, $CBC$ and $CFB$ modes, a
finite number of interleaved sessions, and arbitrary long individual
sessions.    
 


\end{document}